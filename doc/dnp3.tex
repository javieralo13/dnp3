\IfFileExists{t2doc.cls}{
    \documentclass[documentation]{subfiles}
}{
    \errmessage{Error: could not find 't2doc.cls'}
}

\begin{document}

% Declare author and title here, so the main document can reuse it
\trantitle
    {dnp3} % Plugin name
    {dnp3} % Short description
    {Tranalyzer Development Team} % author(s)

\section{dnp3}\label{s:dnp3}

\subsection{Description}
The dnp3 plugin analyzes ...

\subsection{Dependencies}

%\traninput{file} % use this command to input files
%\traninclude{file} % use this command to include files

%\tranimg{image} % use this command to include an image (must be located in a subfolder ./img/)

\subsubsection{External Libraries}
This plugin depends on the {\bf XXX} library.
\begin{table}[!ht]
    \centering
    \begin{tabular}{>{\bf}r>{\tt}l>{\tt}l>{\tt}l}
        \toprule
                                     &                      & {\bf OPT1=1}    & {\bf OPT2=1}\\
        \midrule
        Ubuntu:                      & sudo apt-get install & libXXX-dev      & libYYY-dev\\
        Arch:                        & sudo pacman -S       & libXXX          & YYY\\
        Gentoo:                      & sudo emerge          & libXXX          & YYY\\
        openSUSE:                    & sudo zypper install  & libXXX-devel    & libYYY-devel\\
        Red Hat/Fedora\tablefootnote{If the {\tt dnf} command could not be found, try with {\tt yum} instead}:
                                     & sudo dnf install     & libXXX-devel    & YYY-devel\\
        macOS\tablefootnote{Brew is a packet manager for macOS that can be found here: \url{https://brew.sh}}:
                                     & brew install         & libXXX          & YYY\\
        \bottomrule
    \end{tabular}
\end{table}

\subsubsection{Core Configuration}
This plugin requires the following core configuration:
\begin{itemize}
    \item {\em \$T2HOME/tranalyzer2/src/networkHeaders.h}:
        \begin{itemize}
            \item {\tt ETH\_ACTIVATE>0}
        \end{itemize}
    \item {\em \$T2HOME/tranalyzer2/src/tranalyzer.h}:
        \begin{itemize}
            \item {\tt BLOCK\_BUF=0}
        \end{itemize}
\end{itemize}

\subsubsection{Other Plugins}
This plugin requires the \tranrefpl{tcpFlags} and \tranrefpl{tcpStates} plugins.

\subsubsection{Required Files}
The file {\tt filename.txt} is required.

\subsection{Configuration Flags}
The following flags can be used to control the output of the plugin:
\begin{longtable}{>{\tt}lcl>{\tt\small}l}
    \toprule
    {\bf Name} & {\bf Default} & {\bf Description} & {\bf Flags}\\
    \midrule\endhead%
    %\\
    %\multicolumn{3}{c}{No configuration options available}\\
    %\\
    DNP3\_SAVE     & 0 & Save content to {\tt DNP3\_F\_PATH}                & \\
    DNP3\_RMDIR    & 1 & Empty {\tt DNP3\_F\_PATH} before starting          & DNP3\_SAVE=1\\ % only relevant if DNP3_SAVE=1
    DNP3\_STATS    & 0 & Save statistics to {\tt baseFileName DNP3\_SUFFIX} & \\
    DNP3\_LOAD     & 0 & Load {\tt DNP3\_FNAME}                             & \\
    DNP3\_VAR1     & 0 & Output {\tt dnp3Var1}                              & \\
    DNP3\_IP       & 0 & General description of {\tt DNP3\_IP}              & \\
                      &   & \qquad 0: description of value 0                      & \\
                      &   & \qquad 1: description of value 1                      & \\
                      &   & \qquad 2: description of value 2                      & \\
    DNP3\_VEC      & 0 & Description of {\tt DNP3\_VEC}                     & DNP3\_IP=1   \\ % only relevant if DNP3_IP=1
    DNP3\_ENV\_NUM & 0 & Those variables can be overwritten at runtime         & \\
    DNP3\_ENV\_STR & {\tt\small "str"}
                          & This variable can also be overwritten at runtime      & \\
    DNP3\_FNAME    & {\tt\small "filename.txt"}
                          & File to load                                          & DNP3\_LOAD=1 \\ % only relevant if DNP3_LOAD=1
    DNP3\_SUFFIX   & {\tt\small "\_suffix.txt"}
                          & Suffix for output file                                & DNP3\_STATS=1\\ % only relevant if DNP3_STATS=1
    DNP3\_F\_PATH  & {\tt\small "/tmp/dnp3\_files"}
                          & Suffix for output file                                & DNP3\_STATS=1\\ % only relevant if DNP3_SAVE=1
    \bottomrule
\end{longtable}

\subsubsection{Environment Variable Configuration Flags}
The following configuration flags can also be configured with environment variables ({\tt ENVCNTRL>0}):
\begin{itemize}
    \item {\tt DNP3\_ENV\_NUM}
    \item {\tt DNP3\_ENV\_STR}
\end{itemize}

\subsection{Flow File Output}
The dnp3 plugin outputs the following columns:
\begin{longtable}{>{\tt}lll>{\tt\small}l}
    \toprule
    {\bf Column}                        & {\bf Type} & {\bf Description}                & {\bf Flags}\\
    \midrule\endhead%
    \nameref{dnp3Stat}               & H8         & Status                           & \\
    \hyperref[dnp3Stat]{dnp3Text} & S          & describe dnp3Text (string)    & \\
    \hyperref[dnp3Var1]{dnp3Var1} & U64        & describe dnp3Var1 (uint64)    & DNP3\_VAR1=1\\  % only output if DNP3_VAR1=1
    dnp3IP                           & IP4        & describe dnp3IP (IPv4)        & DNP3\_IP=1  \\  % only output if DNP3_IP=1
    dnp3Var3\_Var4                   & H32\_H16   & describe {\tt dnp3Var3\_Var4} & \\

    \\
    \multicolumn{4}{l}{If {\tt DNP3\_VEC=1}, the following columns are displayed:}\\
    \\

    dnp3Var5\_Var6                   & R(U8\_U8)  & describe {\tt dnp3Var5\_Var6} & \\
    dnp3Vector                       & R(R(D))    & describe {\tt dnp3Vector}     & \\
    \bottomrule
\end{longtable}

\subsubsection{dnp3Stat}\label{dnp3Stat}
The {\tt dnp3Stat} column is to be interpreted as follows:
\begin{longtable}{>{\tt}rl}
    \toprule
    {\bf dnp3Stat} & {\bf Description}\\
    \midrule\endhead%
    0x0\textcolor{magenta}{1} & Flow is dnp3\\
    0x0\textcolor{magenta}{2} & ---\\
    0x0\textcolor{magenta}{4} & ---\\
    0x0\textcolor{magenta}{8} & ---\\
    0x\textcolor{magenta}{1}0 & ---\\
    0x\textcolor{magenta}{2}0 & ---\\
    0x\textcolor{magenta}{4}0 & ---\\
    0x\textcolor{magenta}{8}0 & ---\\
    \bottomrule
\end{longtable}

\subsubsection{dnp3Var1}\label{dnp3Var1}
The {\tt dnp3Var1} column is to be interpreted as follows:\\
\begin{minipage}{.48\textwidth}
    \begin{longtable}{>{\tt}rl}
        \toprule
        {\bf dnp3Var1} & {\bf Description}\\
        \midrule\endhead%
        0x0\textcolor{magenta}{1} & ---\\
        0x0\textcolor{magenta}{2} & ---\\
        0x0\textcolor{magenta}{4} & ---\\
        0x0\textcolor{magenta}{8} & ---\\
        \bottomrule
    \end{longtable}
\end{minipage}
\hfill
\begin{minipage}{.48\textwidth}
    \begin{longtable}{>{\tt}rl}
        \toprule
        {\bf dnp3Var1} & {\bf Description}\\
        \midrule\endhead%
        0x\textcolor{magenta}{1}0 & ---\\
        0x\textcolor{magenta}{2}0 & ---\\
        0x\textcolor{magenta}{4}0 & ---\\
        0x\textcolor{magenta}{8}0 & ---\\
        \bottomrule
    \end{longtable}
\end{minipage}

\subsection{Packet File Output}
In packet mode ({\tt --s} option), the dnp3 plugin outputs the following columns:
\begin{longtable}{>{\tt}lll>{\tt\small}l}
    \toprule
    {\bf Column} & {\bf Type} & {\bf Description} & {\bf Flags}\\
    \midrule\endhead%
    dnp3Col1 & I8 & describe col1 & \\
    \bottomrule
\end{longtable}

\subsection{Monitoring Output}
In monitoring mode, the dnp3 plugin outputs the following columns:
\begin{longtable}{>{\tt}lll>{\tt\small}l}
    \toprule
    {\bf Column} & {\bf Type} & {\bf Description} & {\bf Flags}\\
    \midrule\endhead%
    dnp3Col1 & I8 & describe col1 & \\
    \bottomrule
\end{longtable}

\subsection{Plugin Report Output}
The following information is reported:
\begin{itemize}
    \item Aggregated {\tt\nameref{dnp3Stat}}
    \item Number of XXX packets
\end{itemize}

\subsection{Additional Output}
Non-standard output:
\begin{itemize}
    \item {\tt PREFIX\_suffix.txt}: description
\end{itemize}

\subsection{Post-Processing}

\subsection{Example Output}

\subsection{Known Bugs and Limitations}

\subsection{TODO}
\begin{itemize}
    \item TODO1
    \item TODO2
\end{itemize}

\subsection{References}
\begin{itemize}
    \item \href{https://tools.ietf.org/html/rfcXXXX}{RFCXXXX}: Title
    \item \url{https://www.iana.org/assignments/}
\end{itemize}

\end{document}
